\documentclass{article}
\usepackage{amsmath, amssymb, amsfonts}
\usepackage{fullpage}
\usepackage{enumerate}
\usepackage[linesnumbered,ruled,vlined]{algorithm2e}
\usepackage[usenames,dvipsnames]{xcolor}
\usepackage[colorlinks,allcolors=RoyalBlue]{hyperref}
\usepackage{enumitem}
\usepackage{graphicx} % Required for inserting images
\usepackage[margin=1in]{geometry}
\usepackage{xcolor}
\usepackage{amsmath}
\usepackage{tikz}
\usepackage{tkz-base}
\usepackage{tkz-euclide}
\usepackage{comment}

\usepackage[T1]{fontenc}
% \usepackage{palatino}
\usepackage{libertine, libertinust1math}
\usepackage{titlesec}
% \usepackage{newtxtext,newtxmath}
% \usepackage{kpfonts}

\renewcommand{\familydefault}{\sfdefault}
\renewcommand{\sfdefault}{LinuxLibertineT-TLF}


% For environments and alias
\usepackage{amsthm}
\usepackage{mdframed}
\usepackage{mathtools}
\usepackage{cleveref}
\newmdtheoremenv{assumption}{Assumption}

\DeclareMathOperator*{\argmin}{arg\,min}
\DeclareMathOperator*{\argmax}{arg\,max}
\newcommand{\loss}[1]{\mathcal{L}_\text{#1}}

\newcommand{\R}{\mathbb{R}}
\newcommand{\C}{\mathbb{C}}
\newcommand{\Z}{\mathbb{Z}}
\newcommand{\N}{\mathbb{N}}
\newcommand{\F}{\mathbb{F}}
\newcommand{\E}{\mathbb{E}}
\newcommand{\cO}{\mathcal{O}}
\newcommand{\ctO}{\Tilde{\cO}}
\newcommand{\mdot}{\:\cdot\:}
\newcommand{\jacf}{\nabla f}
\newcommand{\hesf}{\nabla^2 f}

\newcommand{\x}{\mathbf{x}}
\newcommand{\y}{\mathbf{y}}
\newcommand{\w}{\mathbf{w}}

\DeclarePairedDelimiter{\norm}{\lVert}{\rVert}
\DeclarePairedDelimiterX{\inner}[2]{\langle}{\rangle}{#1,#2}
\DeclarePairedDelimiter{\Set}{\lbrace}{\rbrace}
\DeclarePairedDelimiter{\abs}{|}{|}
\DeclarePairedDelimiter{\ceil}{\lceil}{\rceil}
\DeclarePairedDelimiter{\floor}{\lfloor}{\rfloor}
\DeclarePairedDelimiter{\paren}{(}{)}
\DeclarePairedDelimiter{\bracket}{[}{]}
\DeclarePairedDelimiter{\curly}{\lbrace}{\rbrace}
\DeclarePairedDelimiter{\card}{|}{|}
\DeclarePairedDelimiter{\norms}{\lVert}{\rVert^2}
\DeclarePairedDelimiter{\fnorm}{\lVert}{\rVert_F}
\DeclarePairedDelimiter{\infnorm}{\lVert}{\rVert_\infty}

\newcommand{\mincurly}[1]{\min\curly*{#1}}
\newcommand{\maxcurly}[1]{\max\curly*{#1}}
\newcommand{\Exv}[1]{\E\bracket*{#1}}
\newcommand{\bigO}[1]{\cO\paren*{#1}}
\newcommand{\bigtO}[1]{\ctO\paren*{#1}}
\newcommand{\prob}[1]{P\paren*{#1}}
\newcommand{\condprob}[2]{P\paren*{#1\mid#2}}

\newenvironment{solution}{
    \color{Red}
    \textbf{Solution:}
}

\title{CS 4782 Coding Assignment 4 Written Responses\vspace{-10pt}}
\author{Due: 4/24/25 11:59 PM on Gradescope}
\date{Late submissions accepted until 4/26/25 11:59 PM}

\begin{document}
    \maketitle
    \textbf{Note: For homework, you can work in teams of up to 2 people. Please include your teammates’ NetIDs and names on the front page and form a group on Gradescope. (Please answer all questions within each paragraph.) Please show all the relevant steps in your solutions. }
\maketitle
\section*{Problem 1:}

Do you see all 10 digits? Do you notice that some digits are better quality than others? How would you try to improve the quality of the digits?

\section*{Problem 2:}

Do you notice that clusters in the center are smaller than the peripheral clusters on average? If so, why do you think is the case?

\section*{Problem 3:}

What are the sources of stochasticity in the DDIM sampler? How do different samples from the same initial draw of noise relate to each other, if at all? Write 2-3 sentences below.

\section*{Problem 4:}

In section 2.3, the diffusion architecture accepts the noisy latent and the timestep and predicts the added noise. Briefly describe the diffusion architecture being used for this problem? How is the timestep information being incorporated? Write your answer in 3-4 sentences.

\section*{Problem 5:}
Q: What do you observe about the diffusion sampling process? Does there appear to be any relationship between the initial Gaussian latent variable $z_1$ and the final sample from the data distribution $z_0$. Write your answer 3-4 sentences.

\end{document}
